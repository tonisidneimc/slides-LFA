\documentclass[12pt]{article}

% -------------------------------------------------- Dados do discente
% Insira os seus dados e do exercício escolhido:
\def\discente{Fulano de tal}
\def\matricula{20010101}
\def\aa{17}
\def\myling{{99}} % Informe o número da linguagem selecionada.

% -------------------------------------------------- Babel & Geometry
\usepackage[brazil]{babel}
\usepackage[T1]{fontenc}
\usepackage[utf8]{inputenc}
\usepackage[a4paper,top=.5cm,bottom=1cm,left=1cm,right=1cm,includeheadfoot,headheight=52.3318pt,footskip=50pt]{geometry}
%
\usepackage{xcolor}
\usepackage{enumitem}
\usepackage{marvosym} % \Smiley
%
\usepackage{tikz}
\usetikzlibrary{arrows}
\usetikzlibrary{positioning}
%
\usepackage{amsmath,amssymb,amsthm,mathtools}
%
\usepackage{tikz}
\usetikzlibrary{automata,arrows,positioning,shapes}
\tikzset{
 node distance=3cm,
 initial text={$P_{\myling}$},
 double distance=1pt,
 every state/.style={semithick,fill=blue!20!white,minimum size=20pt,inner sep=0pt},
 every edge/.style={draw,->,>=stealth,auto,semithick,font=\ttfamily\small}
}
%
\newcommand{\deriv}[1]{\stackrel{\scriptscriptstyle #1}{\Longrightarrow}}
%

% -------------------------------------------------- Header & Foot
\usepackage{lastpage}
\usepackage{fancyhdr}
\pagestyle{fancy}
\fancyhf{}
\renewcommand{\headrulewidth}{0pt}

\rfoot{\rule{\textwidth}{1.pt}\\\thepage\ de \pageref{LastPage}}

\chead{
 \footnotesize\textbf{Universidade Federal de Goiás -- UFG}\hfill \textsc{Linguagens Formais e Autômatos -- 2021/1}\\
 \footnotesize\textbf{Instituto de Informática -- INF\hfill Prof. Humberto J. Longo} -- \scriptsize\texttt{longo@inf.ufg.br}\\
  \rule{\textwidth}{1.pt}
}

% --------------------------------------------------
\begin{document}
\subsection*{Atividade AA-\aa}
 \paragraph{\matricula -- \discente}
 \begin{itemize}
% 
  \item \textcolor{blue}{Nesta tarefa deve ser escolhida \textbf{uma} das seguintes opções:
   \begin{enumerate}
    \item A partir da gramática livre de contexto proposta na atividade AA-11 (ou de uma versão corrigida da mesma) construa, segundo um dos algoritmos apresentados nas aulas, um PDA que reconheça as cadeias da linguagem gerada pela gramática.
    \item A partir do PDA obtido na atividade ``AA-16 : PDA - Autômatos com pilha'', obtenha uma gramática que gere a linguagem aceita pelo PDA.
   \end{enumerate}}
%- - - - - - - - - - - - - - - - - - - - - - - - - - - - - - - - 
  \item \textcolor{red}{$\mathcal{L}_{\myling} = \{a^nba^n \mid n \geqslant 0\}$.}
%- - - - - - - - - - - - - - - - - - - - - - - - - - - - - - - - 
  \item  \textcolor{red}{Gramática $G_{\myling}$ que gera as cadeias da linguagem $\mathcal{L}_{\myling}$:\\ $G_{\myling}=(V,\Sigma,P,S)=(\{S,B\},\{a,b\},P,S)$, com
  $$
   P =
   \left\{\begin{array}{l}
    S \to aSa \mid B,\\
    B \to b
   \end{array}\right\}.
  $$
  }
%- - - - - - - - - - - - - - - - - - - - - - - - - - - - - - - - 
  \item \textcolor{red}{PDA $P_{\myling}$ que reconhece as cadeias da linguagem $\mathcal{L}_{\myling}$:}
%- - - - - - - - - - - - - - - - - - - - - - - - - - - - - - - -
\begin{center}
\begin{tikzpicture}[
  accept/.style={accepting,fill=green!20!white},
  transform shape,
%  scale=0.95
 ]
  \def\e{\varepsilon}
  \node[initial,state]            (s0) {$s_0$};
  \node[state,accept,right of=s0] (s1) {$s_1$};

  \path (s0) edge [loop above] node {$a,\e\to X$}  (s0)
             edge              node {$b,\e\to \e$} (s1)
        (s1) edge [loop above] node {$a,X\to \e$}  (s1);
\end{tikzpicture}
\end{center}
%- - - - - - - - - - - - - - - - - - - - - - - - - - - - - - - - 
  \item \textcolor{red}{[OPÇÂO 1.1 :] PDA $P^1_{\myling}$, construído a partir da gramática $G_{\myling}$, que reconhece as cadeias da linguagem $\mathcal{L}_{\myling}$:}
   \begin{center}
   \begin{tikzpicture}[
     accept/.style={accepting,fill=green!20!white},
     slopea/.style={sloped,anchor=center,above},
     slopeb/.style={sloped,anchor=center,below},
     transform shape,
   %  scale=0.95
    ]
     \def\e{\varepsilon}
     \node[initial,state]                         (s0) {$s_0$};
     \node[state,right of=s0]                     (s1) {$s_1$};
     \node[state,right of=s1]                     (s2) {$s_2$};
     \node[state,above left of=s2,xshift=15pt]    (s3) {$s_3$};
     \node[state,above right of=s2,xshift=-15pt]  (s4) {$s_4$};
     \node[state,accept,right of=s2]              (s5) {$s_5$};
   
     \path (s0) edge node {$\e,\e\to \$$}            (s1)
           (s1) edge node {$\e,\e\to S$}             (s2)
           (s2) edge node [slopea] {$\e,S\to a$}     (s3)
                edge node {$\e,\$\to \e$}            (s5)
                edge [loop below,align=center]
                     node {$\e,S\to B$\\$\e,B\to b$\\
                           $a,a\to \e$\\$b,b\to \e$} (s2)
           (s3) edge node {$\e,\e\to S$}             (s4)
           (s4) edge node [slopea] {$\e,\e\to a$}    (s2);
   \end{tikzpicture}
   \end{center}
%- - - - - - - - - - - - - - - - - - - - - - - - - - - - - - - -
  \item \textcolor{red}{[OPÇÂO 1.2 :] PDA $P^2_{\myling}$ que reconhece as cadeias da linguagem $\mathcal{L}_{\myling}$, construído a partir da versão na forma normal de Greibach da gramática $G_{\myling}$:}
%- - - - - - - - - - - - - - - - - - - - - - - - - - - - - - - -
   \begin{itemize}
    \item \textcolor{red}{Versão na forma normal de Greibach da gramática $G_{\myling}$:\\ $G^2_{\myling}=(V,\Sigma,P,S)=(\{S_0,S,A\},\{a,b\},P,S)$, com
    $$
     P =
     \left\{\begin{array}{r@{\;\to\;}l}
      S_0 & aSA \mid b,\\
      S   & aSA \mid b,\\
      A   & a
     \end{array}\right\}.
    $$
    }
%- - - - - - - - - - - - - - - - - - - - - - - - - - - - - - - -
    \item \textcolor{red}{PDA $P^2_{\myling}$ obtido a partir da gramática $G^1_{\myling}$:}
     \begin{center}
     \begin{tikzpicture}[
       accept/.style={accepting,fill=green!20!white},
       slopea/.style={sloped,anchor=center,above},
       slopeb/.style={sloped,anchor=center,below},
       transform shape,
     %  scale=0.95
      ]
       \def\e{\varepsilon}
       \node[initial,state]                         (s0) {$s_0$};
       \node[state,right of=s0]                     (s1) {$s_1$};
       \node[state,right of=s1]                     (s2) {$s_2$};
       \node[state,accept,right of=s2]              (s3) {$s_3$};
     
       \path (s0) edge node {$\e,\e\to \$$}                         (s1)
             (s1) edge [align=left]
                       node {$b,\e\to \e$\\$a,\e\to SB$}             (s2)
             (s2) edge [loop below,align=left]
                       node {$a,B\to \e$\\$b,S\to \e$\\$a,S\to SB$} (s2)
             (s2) edge node {$\e,\$\to \e$}                         (s3);
     \end{tikzpicture}
     \end{center}

   \end{itemize}
%   }
%- - - - - - - - - - - - - - - - - - - - - - - - - - - - - - - -
  \item \textcolor{red}{[OPÇÂO 2 :] Gramática $G^1_{\myling}$ obtida a partir do PDA $P_{\myling}$:
   \begin{itemize}
    \item Vide exemplo 7.28 nos slides \ldots {\Large \Smiley}
%- - - - - - - - - - - - - - - - - - - - - - - - - - - - - - - -
   \end{itemize}
}
  \end{itemize}
%- - - - - - - - - - - - - - - - - - - - - - - - - - - - - - - -
%
\noindent\textbf{Obs.:} Esta observação e os comandos de cor (``\verb|\textcolor{}{}|'') podem ser removidos da versão a ser submetida na plataforma Turing.
\end{document}
