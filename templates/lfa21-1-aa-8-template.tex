\documentclass[12pt]{article}

% -------------------------------------------------- Dados do discente
% Insira os seus dados e do exercício escolhido:
\def\discente{Fulano de tal}
\def\matricula{20010101}
\def\aa{8}
\def\myling{{99}} % Informe o número da linguagem selecionada.

% -------------------------------------------------- Babel & Geometry
\usepackage[brazil]{babel}
\usepackage[T1]{fontenc}
\usepackage[utf8]{inputenc}
\usepackage[a4paper,top=.5cm,bottom=1cm,left=1cm,right=1cm,includeheadfoot,headheight=52.3318pt,footskip=50pt]{geometry}

% -------------------------------------------------- TikZ
\usepackage{tikz}
\usetikzlibrary{automata,arrows,positioning}
\tikzset{
 node distance=2.5cm,
 initial text={$M_\myling$},
 double distance=1pt,
 every state/.style={semithick,fill=blue!20!white,minimum size=20pt,inner sep=0pt},
 every edge/.style={draw,->,>=stealth,auto,semithick,font=\ttfamily\small}
}

% -------------------------------------------------- Header & Foot
\usepackage{lastpage}
\usepackage{fancyhdr}
\pagestyle{fancy}
\fancyhf{}
\renewcommand{\headrulewidth}{0pt}

\rfoot{\rule{\textwidth}{1.pt}\\\thepage\ de \pageref{LastPage}}

\chead{
 \footnotesize\textbf{Universidade Federal de Goiás -- UFG}\hfill \textsc{Linguagens Formais e Autômatos -- 2021/1}\\
 \footnotesize\textbf{Instituto de Informática -- INF\hfill Prof. Humberto J. Longo} -- \scriptsize\texttt{longo@inf.ufg.br}\\
  \rule{\textwidth}{1.pt}
}

% --------------------------------------------------
\begin{document}
\subsection*{Atividade AA-\aa}
 \paragraph{\matricula -- \discente}
%
%
 \begin{itemize}
  \item A partir de cada uma das gramáticas obtidas na AA-7, use o algoritmo baseado em sistemas de equações para extrair expressões regulares que gerem as mesmas cadeias geradas pelas gramáticas.
%- - - - - - - - - - - - - - - - - - - - - - - - - - - - - - - - 
  \item \textcolor{blue}{$\mathcal{L}_\myling = \{w\mid w=0u1$ ou $w=1u0$, com $u\in\Sigma^*\}$.}
  
% - - - - - - - - - - - - - - - - - - - - - - - - - - - - - - - - - - - -
  \item  \textcolor{red}{Gramática $G_1$ que gera as cadeias da linguagem $\mathcal{L}_{\myling}$:\\ $G_1=(V,\Sigma,P,S)=(\{A,B,C,D,S\},\{0,1\},P,S)$, com
  $$
   P =
   \left\{\begin{array}{l}
    S \to 0A \mid 1C,\\
    A \to 0A \mid 1B,\\
    B \to 0A \mid 1B \mid \varepsilon,\\
    C \to 0D \mid 1C,\\
    D \to 0D \mid 1C \mid \varepsilon
   \end{array}\right\}.
  $$
  }
%- - - - - - - - - - - - - - - - - - - - - - - - - - - - - - - - 
  \item  \textcolor{red}{Extração de expressão regular $\mathcal{R}_1$ da gramática $G$, tal que $\mathcal{L}(\mathcal{R}_1)=\mathcal{L}(G_1)$:\\
  $$
  \begin{array}{ccll}
  \hline
  \textbf{Etapa} & \textbf{Item} & \textbf{Expressão} & \textbf{Ação}\\
  \hline
    I & 1 & S = 0A \cup 1C &\\
      & 2 & A = 0A \cup 1B &\\
      & 3 & B = 0A \cup 1B \cup \varepsilon&\\
      & 4 & C = 0D \cup 1C &\\
      & 5 & D = 0D \cup 1C \cup \varepsilon&\\
  \hline
   II & 1 & S = 0A \cup 1C &\\
      & 2 & A = 0^*1B      & I.2 \rightarrow \texttt{Lema de Arden}\\
      & 3 & B = 0A \cup 1B \cup \varepsilon&\\
      & 4 & C = 1^*0D      & I.4 \rightarrow \texttt{Lema de Arden}\\
      & 5 & D = 0D \cup 1C \cup \varepsilon&\\
  \hline
  III & 1 & S = 0^+1B \cup 1^+0D & II.2, II.2 \rightarrow II.1\\
      & 3 & B = 0^+1B \cup 1B \cup \varepsilon & II.2 \rightarrow II.3\\
      & 5 & D = 0D \cup 1^+0D \cup \varepsilon&II.4 \rightarrow II.5\\\\
  \hline
   IV & 1 & S = 0^+1B \cup 1^+0D &\\
      & 3 & B = (0^+1 \cup 1)^* & III.3 \rightarrow \texttt{Lema de Arden}\\
      & 5 & D = (1^+0 \cup 0)^* & III.5 \rightarrow \texttt{Lema de Arden}\\
  \hline
    V & 1 & S = 0^+1(0^+1 \cup 1)^* \cup 1^+0(1^+0 \cup 0)^* & IV.3, IV.5 \rightarrow IV.1\\
  \hline
  \end{array}
  $$
  }
%- - - - - - - - - - - - - - - - - - - - - - - - - - - - - - - - 
  \item  \textcolor{red}{$\mathcal{R}_1= 0^+1(0^+1 \cup 1)^* \cup 1^+0(1^+0 \cup 0)^*$.}
%- - - - - - - - - - - - - - - - - - - - - - - - - - - - - - - - 
  \newpage
% - - - - - - - - - - - - - - - - - - - - - - - - - - - - - - - - - - - -
% - - - - - - - - - - - - - - - - - - - - - - - - - - - - - - - - - - - -
  \item  \textcolor{red}{Gramática $G_2$ que gera as cadeias da linguagem $\mathcal{L}_{\myling}$:\\ $G_2=(V,\Sigma,P,S)=(\{A,B,C,D,S\},\{0,1\},P,S)$, com
  $$
   P =
   \left\{\begin{array}{l}
    S \to A \mid E,\\
    A \to 0B,\\
    B \to C \mid 0C \mid 1C\\
    C \to B \mid 1D,\\
    D \to I\\
    E \to 1F,\\
    F \to G \mid 0G \mid 1G\\
    G \to F \mid 0H,\\
    H \to I\\
    I \to \varepsilon
   \end{array}\right\}.
  $$
  }
%- - - - - - - - - - - - - - - - - - - - - - - - - - - - - - - - 
  \item  \textcolor{red}{Extração de expressão regular $\mathcal{R}_2$ da gramática $G_2$, tal que $\mathcal{L}(\mathcal{R}_2)=\mathcal{L}(G_2)$:\\
  $$
  \begin{array}{ccll}
  \hline
  \textbf{Etapa} & \textbf{Item} & \textbf{Expressão} & \textbf{Ação}\\
  \hline
    I & 1 & S = A \cup E &\\
      & 2 & A = 0B &\\
      & 3 & B = C \cup 0C \cup 1C&\\
      & 4 & C = B \cup 1D &\\
      & 5 & D = I&\\
      & 6 & E = 1F&\\
      & 7 & F = G \cup 0G \cup 1G&\\
      & 8 & G = F \cup 0H&\\
      & 9 & H = I&\\
      & 10& I = \varepsilon&\\
  \hline
   II & 1 & S = A \cup E &\\
      & 2 & A = 0B &\\
      & 3 & B = C \cup 0C \cup 1C&\\
      & 4 & C = B \cup 1 & I.10\rightarrow I.5\rightarrow I.4\\
      & 6 & E = 1F&\\
      & 7 & F = G \cup 0G \cup 1G&\\
      & 8 & G = F \cup 0 & I.10\rightarrow I.9\rightarrow I.8\\
  \hline
  III & 1 & S = A \cup E &\\
      & 2 & A = 0B &\\
      & 3 & B = B \cup 1 \cup 0B \cup 01 \cup 1B \cup 11 & II.4\rightarrow II.3\\
      & 6 & E = 1F&\\
      & 7 & F = F \cup 0 \cup 0F \cup 00 \cup 1F \cup 10& II.8\rightarrow II.7\\
  \hline
   IV & 1 & S = A \cup E &\\
      & 2 & A = 0B &\\
      & 3 & B = (\varepsilon \cup 0 \cup 1)B \cup (\varepsilon \cup 0 \cup 1)1 & III.3 \rightarrow\texttt{Fatoração.}\\
      & 6 & E = 1F&\\
      & 7 & F = (\varepsilon \cup 0 \cup 1)F \cup (\varepsilon \cup 0 \cup 1)0 & III.7 \rightarrow\texttt{Fatoração.}\\
  \hline
    V & 1 & S = A \cup E &\\
      & 2 & A = 0B &\\
      & 3 & B = (\varepsilon \cup 0 \cup 1)^*(\varepsilon \cup 0 \cup 1)1 & IV.3 \rightarrow\texttt{Lema de Arden.}\\
      & 6 & E = 1F&\\
      & 7 & F = (\varepsilon \cup 0 \cup 1)^*(\varepsilon \cup 0 \cup 1)0 & IV.7 \rightarrow\texttt{Lema de Arden.}\\
  \hline
   VI & 1 & S = A \cup E &\\
      & 2 & A = 0(\varepsilon \cup 0 \cup 1)^*(\varepsilon \cup 0 \cup 1)1 & V.3 \rightarrow V.2\\
      & 6 & E = 1(\varepsilon \cup 0 \cup 1)^*(\varepsilon \cup 0 \cup 1)0& V.7 \rightarrow V.6\\
  \hline
  VII & 1 & S = 0(\varepsilon \cup 0 \cup 1)^*(\varepsilon \cup 0 \cup 1)1\ \cup & VI.2, VI.6 \rightarrow VI.1\\
      &   & \phantom{ A =}\ 1(\varepsilon \cup 0 \cup 1)^*(\varepsilon \cup 0 \cup 1)0\\
  \hline
  \end{array}
  $$
  }
%- - - - - - - - - - - - - - - - - - - - - - - - - - - - - - - - 
  \item  \textcolor{red}{$\mathcal{R}_2 = 0(\varepsilon \cup 0 \cup 1)^*(\varepsilon \cup 0 \cup 1)1 \cup 1(\varepsilon \cup 0 \cup 1)^*(\varepsilon \cup 0 \cup 1)0$.}
%- - - - - - - - - - - - - - - - - - - - - - - - - - - - - - - - 
\end{itemize}
%

\noindent\textbf{Obs.:} A expressão regular obtida ao final desse procedimento pode variar, a depender da sequência de substituições de variáveis e da aplicação do Lema de Arden. Esta observação e os comandos de cor (``\verb|\textcolor{}{}|'') podem ser removidos da versão a ser submetida na plataforma Turing.
\end{document}
