\documentclass[12pt]{article}

% -------------------------------------------------- Dados do discente
% Insira os seus dados e do exercício escolhido:
\def\discente{Fulano de tal}
\def\matricula{20010101}
\def\aa{10}
\def\myling{{99}} % Informe o número da linguagem selecionada.

% -------------------------------------------------- Babel & Geometry
\usepackage[brazil]{babel}
\usepackage[T1]{fontenc}
\usepackage[utf8]{inputenc}
\usepackage[a4paper,top=.5cm,bottom=1cm,left=1cm,right=1cm,includeheadfoot,headheight=52.3318pt,footskip=50pt]{geometry}

\usepackage{xcolor}
\usepackage{amsmath,amssymb,amsthm}
\newtheorem{lema}{Lema}

% -------------------------------------------------- Header & Foot
\usepackage{lastpage}
\usepackage{fancyhdr}
\pagestyle{fancy}
\fancyhf{}
\renewcommand{\headrulewidth}{0pt}

\rfoot{\rule{\textwidth}{1.pt}\\\thepage\ de \pageref{LastPage}}

\chead{
 \footnotesize\textbf{Universidade Federal de Goiás -- UFG}\hfill \textsc{Linguagens Formais e Autômatos -- 2021/1}\\
 \footnotesize\textbf{Instituto de Informática -- INF\hfill Prof. Humberto J. Longo} -- \scriptsize\texttt{longo@inf.ufg.br}\\
  \rule{\textwidth}{1.pt}
}

% --------------------------------------------------
\begin{document}
\subsection*{Atividade AA-\aa}
 \paragraph{\matricula -- \discente}
%
 \begin{itemize}
  \item \textcolor{red}{$\mathcal{L}_{\myling} = \{w\in\{0,1\}^*\mid w = 0^n10^n,\ n \geqslant 0\}$.}
\end{itemize}
%
\textcolor{blue}{
\begin{lema}
A linguagem $\mathcal{L}_{\myling}$ não é regular.
\end{lema}
}
%
\begin{proof}
\textcolor{blue}{Suponha que $\mathcal{L}_{\myling}$ seja regular. Neste caso,  $\mathcal{L}_{\myling}$ é reconhecida por um autômato finito determinístico com $k$ estados. O \emph{Pumping Lemma} para linguagens regulares garante que qualquer cadeia $w\in  \mathcal{L}_{\myling}$, tal que $|w|\geqslant k$, pode ser subdividida em subcadeias $x$, $y$ e $z$ satisfazendo $w=xyz$, $|xy|\leqslant k$, $|y|>0$ ($y\neq\varepsilon$) e $xy^iz\in \mathcal{L}_{\myling}$, para $i\geqslant 0$.}

\textcolor{red}{Contudo, considere a cadeia $w=xyz=0^k10^k\in \mathcal{L}_{\myling}$. Segundo o \emph{Pumping Lemma} para linguagens regulares $|xy|\leqslant k$, ou seja, $z=0^{k-|xy|}10^k$. Assim,
\begin{align*}
 w' = xy^2z &= (xy)(y)(z)\\
            &= (0^{|xy|})(0^{|y|})(0^{k-|xy|}10^k)\\
            &= 0^{k+|y|}10^k.
\end{align*}}
\textcolor{blue}{Logo, $w'\notin\mathcal{L}_{\myling}$, o que contradiz o \emph{Pumping Lemma}. Portanto, é falsa a suposição de que $\mathcal{L}_{\myling}$ é regular. }
\end{proof}
%
\noindent\textbf{Obs.:} O texto na cor azul é praticamente padrão, no sentido de que contém as informações básicas necessárias, e não deve variar muito ao se considerar outras linguagens. Contudo, essas informações podem ser organizadas de diferentes outras formas. Já o texto em vermelho pode variar bastante, a depender da linguagem que está sendo considerada. Esta observação e os comandos de cor (``\verb|\textcolor{}{}|'') podem ser removidos da versão a ser submetida na plataforma Turing.
\end{document}
