\documentclass[12pt]{article}

% --------------------------------------------------
% Insira os seus dados e do exercício escolhido:
\def\discente{Fulano de tal}
\def\matricula{20010101}
\def\aa{4}
\def\myling{{99}} % Informe o número da linguagem selecionada.
% --------------------------------------------------

\usepackage[brazil]{babel}
\usepackage[T1]{fontenc}
\usepackage[utf8]{inputenc}
\usepackage[a4paper,top=1.5cm,bottom=1.5cm,left=2.0cm,right=1.5cm,nohead,nofoot]{geometry}

\usepackage{tikz}
\usetikzlibrary{automata,arrows,positioning}
\tikzset{
 node distance=2.5cm,
 initial text={$M_\myling$},
 double distance=1pt,
 every state/.style={semithick,fill=blue!20!white,minimum size=20pt,inner sep=0pt},
 every edge/.style={draw,->,>=stealth,auto,semithick,font=\ttfamily\small}
}


\begin{document}
\subsection*{Atividade AA-\aa}
 \paragraph{\matricula -- \discente}
%
 \begin{itemize}
  \item \textcolor{blue}{$\mathcal{L}_\myling = \{w\mid w$ contém $010$ exatamente uma vez$\}$.}
%  
  \item  \textcolor{red}{Autômato finito determinístico que reconhece as cadeias $\mathcal{L}_\myling$:\\
  $M_\myling=\langle\Sigma=\{0,1\},S=\{s_0,s_1,s_2,s_3,s_4,s_5\},s_0,\delta,F=\{s_3,s_4\}\rangle$, com a função $\delta$ definida por:\\
  $$\begin{array}{|c|cc|}
   \hline
   \delta & 0   & 1\\
   \hline
      s_0 & s_1 & s_0\\
      s_1 & s_1 & s_2\\
      s_2 & s_3 & s_0\\
      s_3 & s_3 & s_4\\
      s_4 & s_5 & s_3\\
      s_5 & s_5 & s_5\\
   \hline
  \end{array}$$}


 \end{itemize}
%
\begin{center}
 \begin{tikzpicture}[%node distance=2.cm,
    transform shape,
    scale=1.3
   ]
    \node[state,initial]                                   (s0) {$s_0$};
    \node[state,right of=s0]                               (s1) {$s_1$};
    \node[state,right of=s1]                               (s2) {$s_2$};
    \node[state,fill=green!20!white,accepting,right of=s2] (s3) {$s_3$};
    \node[state,fill=green!20!white,accepting,below of=s3] (s4) {$s_4$};
    \node[state,fill=red!20!white,left of=s4]              (s5) {$s_5$};
  %
    \draw (s0) edge                               node {0}   (s1)
               edge[loop above]                   node {1}   (s0)
          (s1) edge[loop above]                   node {0}   (s1)
               edge                               node {1}   (s2)
          (s2) edge                               node {0}   (s3)
%               edge[out=210,in=320,looseness=0.5] node {1}   (s0)
               edge[bend left=30]                 node {1}   (s0)
          (s3) edge[loop above]                   node {0}   (s3)
               edge[bend left=20]                 node {1}   (s4)
          (s4) edge                               node {0}   (s5)
               edge[bend left=20]                 node {1}   (s3)
          (s5) edge[loop above]                   node {0,1} (s5);
   \end{tikzpicture}
\end{center}
%

\noindent\textbf{Obs.:} esta observação e os comandos de cor (``\verb|\textcolor{}{}|'') podem ser removidos da versão a ser submetida na plataforma Turing.
\end{document}
