\documentclass[12pt]{article}

% -------------------------------------------------- Dados do discente
% Insira os seus dados e do exercício escolhido:
\def\discente{Fulano de tal}
\def\matricula{20010101}
\def\aa{15}
\def\myling{{99}} % Informe o número da linguagem selecionada.

% -------------------------------------------------- Babel & Geometry
\usepackage[brazil]{babel}
\usepackage[T1]{fontenc}
\usepackage[utf8]{inputenc}
\usepackage[a4paper,top=.5cm,bottom=1cm,left=1cm,right=1cm,includeheadfoot,headheight=52.3318pt,footskip=50pt]{geometry}
%
\usepackage{xcolor}
\usepackage{enumitem}
%
\usepackage{tikz}
\usetikzlibrary{arrows}
\usetikzlibrary{positioning}
%
\usepackage{amsmath,amssymb,amsthm,mathtools}
%
\newcommand{\deriv}[1]{\stackrel{\scriptscriptstyle #1}{\Longrightarrow}}
%

% -------------------------------------------------- Header & Foot
\usepackage{lastpage}
\usepackage{fancyhdr}
\pagestyle{fancy}
\fancyhf{}
\renewcommand{\headrulewidth}{0pt}

\rfoot{\rule{\textwidth}{1.pt}\\\thepage\ de \pageref{LastPage}}

\chead{
 \footnotesize\textbf{Universidade Federal de Goiás -- UFG}\hfill \textsc{Linguagens Formais e Autômatos -- 2021/1}\\
 \footnotesize\textbf{Instituto de Informática -- INF\hfill Prof. Humberto J. Longo} -- \scriptsize\texttt{longo@inf.ufg.br}\\
  \rule{\textwidth}{1.pt}
}

% --------------------------------------------------
\begin{document}
\subsection*{Atividade AA-\aa}
 \paragraph{\matricula -- \discente}
%
 \begin{itemize}
%- - - - - - - - - - - - - - - - - - - - - - - - - - - - - - - - 
  \item Converta a gramática $G_5$, obtida na AA-14, para a forma normal de Chomsky \textbf{OU} para a forma normal de Greibach.
  
%- - - - - - - - - - - - - - - - - - - - - - - - - - - - - - - - 
  \item \textcolor{red}{$\mathcal{L}_{\myling} = \{a^nb^{2m+1}c^{2m+1}a^{2n} \mid n,m \geqslant 0\}$.}
%- - - - - - - - - - - - - - - - - - - - - - - - - - - - - - - - 
  \item  \textcolor{red}{Gramática $G_{\myling}^5$ (obtida na AA-14) que gera as cadeias da linguagem $\mathcal{L}_{\myling}$:\\
  $G_{\myling}^5=(V,\Sigma,P,S_0)=(\{A,S,S_0\},\{a,b,c\},P,S_0)$, com
    $$
     P =
     \left\{\begin{array}{l}
      S_0\to aSaa\mid bbAcc\mid bc\\
      S\to aSaa\mid bbAcc\mid bc\\
      A\to bbAcc\mid bc
     \end{array}\right\}.
    $$
  }
\end{itemize}
%- - - - - - - - - - - - - - - - - - - - - - - - - - - - - - - -
\textcolor{blue}{
\begin{enumerate}
  \item Gramática na forma normal de Chomsky, obtida a partir de $G_{\myling}^5$,  que gera as cadeias da linguagem $\mathcal{L}_{\myling}$:\\
  $G_{\myling}^6=(V,\Sigma,P,S_0)=(\{A,S,S_0,T_1,T_2,T_3,T_4,T_5,X,Y,Z\},\{a,b,c\},P,S_0)$, com
    $$
     P =
     \left\{\begin{array}{l}
      S_0\to XT_1\mid YT_2\mid YZ\\
      S\to XT_1\mid YT_2\mid YZ\\
      A\to YT_2\mid YZ\\
      T_1\to ST_3\\
      T_2\to YT_4\\
      T_3\to XX\\
      T_4\to AT_5\\
      T_5\to ZZ\\
      X\to a\\
      Y\to b\\
      Z\to c
     \end{array}\right\}.
    $$
%- - - - - - - - - - - - - - - - - - - - - - - - - - - - - - - -
  \item Gramática intermediária $G_{\myling}^7$, obtida de $G_{\myling}^6$, definindo-se números de ordem para as variáveis e fazendo-se as substituições apropriadas nas regras de derivação das variáveis $T_1$ e $T_4$:\\
  $G_{\myling}^7=(V,\Sigma,P,S_0)=(\{A,S,S_0,T_1,T_2,T_3,T_4,T_5,X,Y,Z\},\{a,b,c\},P,S_0)$, com
    $$
     P =
     \left\{\begin{array}{l}
      S_0\to XT_1\mid YT_2\mid YZ\\
      S\to XT_1\mid YT_2\mid YZ\\
      A\to YT_2\mid YZ\\
      T_1\to XT_1T_3\mid YT_2T_3\mid YZT_3\\
      T_2\to YT_4\\
      T_3\to XX\\
      T_4\to YT_2T_5\mid YZT_5\\
      T_5\to ZZ\\
      X\to a\\
      Y\to b\\
      Z\to c
     \end{array}\right\}.
    $$
    $$
     \begin{array}{l|ccccccccccc}
      \textbf{Variável} & S_0 & S & A & T_1 & T_2 & T_3 & T_4 & T_5 & X & Y & Z\\
      \hline
      \textbf{Ordem} & 1 & 2 & 3 & 4 & 5 & 6 & 7 & 8 & 9 & 10 & 11\\
     \end{array}
    $$
%- - - - - - - - - - - - - - - - - - - - - - - - - - - - - - - -
%- - - - - - - - - - - - - - - - - - - - - - - - - - - - - - - -
  \item Gramática $G_{\myling}^8$ na forma normal de Greibach, obtida de $G_{\myling}^7$, substituindo-se as variáveis $X$, $Y$ e $Z$ mais à esquerda nas regras de derivação:\\
  $G_{\myling}^8=(V,\Sigma,P,S_0)=(\{A,S,S_0,T_1,T_2,T_3,T_4,T_5,X,Y,Z\},\{a,b,c\},P,S_0)$, com
    $$
     P =
     \left\{\begin{array}{l}
      S_0\to aT_1\mid bT_2\mid bZ\\
      S\to aT_1\mid bT_2\mid bZ\\
      A\to bT_2\mid bZ\\
      T_1\to aT_1T_3\mid bT_2T_3\mid bZT_3\\
      T_2\to bT_4\\
      T_3\to aX\\
      T_4\to bT_2T_5\mid bZT_5\\
      T_5\to cZ\\
      X\to a\\
      Y\to b\\
      Z\to c
     \end{array}\right\}.
    $$
\end{enumerate}
}
%- - - - - - - - - - - - - - - - - - - - - - - - - - - - - - - -
%
\noindent\textbf{Obs.:} Esta observação e os comandos de cor (``\verb|\textcolor{}{}|'') podem ser removidos da versão a ser submetida na plataforma Turing.
\end{document}
