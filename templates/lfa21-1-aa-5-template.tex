\documentclass[12pt]{article}

% --------------------------------------------------
% Insira os seus dados e do exercício escolhido:
\def\discente{Fulano de tal}
\def\matricula{20010101}
\def\aa{5}
\def\myling{{99}} % Informe o número da linguagem selecionada.
% --------------------------------------------------

\usepackage[brazil]{babel}
\usepackage[T1]{fontenc}
\usepackage[utf8]{inputenc}
\usepackage[a4paper,top=1.0cm,bottom=1.0cm,left=1.0cm,right=1.0cm,nohead,nofoot]{geometry}

\usepackage{amssymb}


\usepackage{tikz}
\usetikzlibrary{automata,arrows,positioning,shapes}
\tikzset{
 node distance=2.5cm,
% initial text={$M_{\myling}$},
 double distance=1pt,
 every state/.style={semithick,fill=blue!20!white,minimum size=20pt,inner sep=0pt},
 every edge/.style={draw,->,>=stealth,auto,semithick,font=\ttfamily\small}
}


\begin{document}
\pagestyle{empty}
\subsection*{Atividade AA-\aa}
 \paragraph{\matricula -- \discente}
%
 \begin{itemize}
%- - - - - - - - - - - - - - - - - - - - - - - - - - - - - - - 
  \item Construa o DFA (\emph{tupla} e diagrama de estados) resultante do produto $\otimes$ do DFA proposto na AA-4 com o DFA mínimo que reconhece a linguagem $\mathcal{L}_{par}=\{w\in\{0,1\}^*\mid |w|=2\cdot k,\ k\in\mathbb{N}\}$. \textbf{Obs.:} Caso a DFA proposto na AA-4 tenha sido avaliado como incorreto, use uma versão corrigida do mesmo.
%- - - - - - - - - - - - - - - - - - - - - - - - - - - - - - - - 
  \item  \textcolor{blue}{Autômato finito determinístico que reconhece as cadeias pertencentes à linguagem $\mathcal{L}_{par}$:}
   \begin{center}
   \begin{tikzpicture}[
    node distance=2.cm,
    initial text={$M_{\mathcal{L}_{par}}$},
%    transform shape,
%    scale=1.2
   ]
    \node[state,initial,fill=green!20!white,accepting] (q0) {$q_0$};
    \node[state,right of=q0]                           (q1) {$q_1$};
  %
    \draw (q0) edge[bend left=20] node {0,1} (q1)
          (q1) edge[bend left=20] node {0,1} (q0);
   \end{tikzpicture}
  \end{center}
%- - - - - - - - - - - - - - - - - - - - - - - - - - - - - - - - 
  \item \textcolor{red}{$\mathcal{L}_\myling = \{w\mid w$ contém $010$ exatamente uma vez$\}$. (deve ser a linguagem selecionada por cada um)}
%- - - - - - - - - - - - - - - - - - - - - - - - - - - - - - - - 
  \item  \textcolor{red}{Autômato finito determinístico que reconhece as cadeias da linguagem  $\mathcal{L}_\myling$: (deve ser o DFA que foi proposto na AA-4)}
  \begin{center}
   \begin{tikzpicture}[
    node distance=2.cm,
    initial text={$M_\myling$},
%    transform shape,
%    scale=1.2
   ]
    \node[state,initial]                                   (s0) {$s_0$};
    \node[state,right of=s0]                               (s1) {$s_1$};
    \node[state,right of=s1]                               (s2) {$s_2$};
    \node[state,fill=green!20!white,accepting,right of=s2] (s3) {$s_3$};
    \node[state,fill=green!20!white,accepting,right of=s3] (s4) {$s_4$};
    \node[state,fill=red!20!white,right of=s4]             (s5) {$s_5$};
  %
    \draw (s0) edge               node {0}   (s1)
               edge[loop above]   node {1}   (s0)
          (s1) edge[loop above]   node {0}   (s1)
               edge               node {1}   (s2)
          (s2) edge               node {0}   (s3)
               edge[bend left=30] node {1}   (s0)
          (s3) edge[loop above]   node {0}   (s3)
               edge[bend left=20] node {1}   (s4)
          (s4) edge[bend left=20] node {1}   (s3)
               edge               node {0}   (s5)
          (s5) edge[loop above]   node {0,1} (s5);
   \end{tikzpicture}
\end{center}
%- - - - - - - - - - - - - - - - - - - - - - - - - - - - - - - - 
  \item  \textcolor{red}{Autômato finito determinístico que reconhece as cadeias da linguagem $\mathcal{L}(\mathcal{M}_{\mathcal{L}_{par}} \otimes \mathcal{M}_\myling) \equiv \mathcal{L}_{par} \cap \mathcal{L}_\myling \equiv \{w\mid |w|$ é par e $w$ contém $010$ exatamente uma vez$\}: D = \langle \Sigma=\{0,1\}, S=\{s_0{\cdot}q_0,s_0{\cdot}q_1,s_1{\cdot}q_0,\allowbreak s_1{\cdot}q_1,s_2{\cdot}q_0,s_2{\cdot}q_1,s_3{\cdot}q_0,s_3{\cdot}q_1,s_4{\cdot}q_0,s_4{\cdot}q_1,s_5{\cdot}q_0,s_5{\cdot}q_1\},s_0{\cdot}q_0, \delta, F=\{s_3{\cdot}q_0,s_4{\cdot}q_0,\rangle$, com a função $\delta$ definida por:}
    $$\begin{array}{|c|cc|}
     \hline
     \delta        & 0             & 1\\
     \hline
     s_0{\cdot}q_0 & s_1{\cdot}q_1 & s_0{\cdot}q_1\\
     s_0{\cdot}q_1 & s_1{\cdot}q_0 & s_0{\cdot}q_0\\
     s_1{\cdot}q_0 & s_1{\cdot}q_1 & s_2{\cdot}q_1\\
     s_1{\cdot}q_1 & s_1{\cdot}q_0 & s_2{\cdot}q_0\\
     s_2{\cdot}q_0 & s_3{\cdot}q_1 & s_0{\cdot}q_1\\
     s_2{\cdot}q_1 & s_3{\cdot}q_0 & s_0{\cdot}q_0\\
     s_3{\cdot}q_0 & s_3{\cdot}q_1 & s_4{\cdot}q_1\\
     s_3{\cdot}q_1 & s_3{\cdot}q_0 & s_4{\cdot}q_0\\
     s_4{\cdot}q_0 & s_5{\cdot}q_1 & s_3{\cdot}q_1\\
     s_4{\cdot}q_1 & s_5{\cdot}q_0 & s_3{\cdot}q_0\\
     s_5{\cdot}q_0 & s_5{\cdot}q_1 & s_5{\cdot}q_1\\
     s_5{\cdot}q_1 & s_5{\cdot}q_0 & s_5{\cdot}q_0\\
     \hline
    \end{array}$$
  
  \begin{center}
   \begin{tikzpicture}[
%    node distance=3.cm,
    initial text={$D$},
%    transform shape,
%    scale=1.2
   ]
    \node[state,initial above,ellipse]                              (p00) {$s_0{\cdot}q_0$};
    \node[state,right of=p00,ellipse]                               (p11) {$s_1{\cdot}q_1$};
    \node[state,above of=p11,ellipse]                               (p20) {$s_2{\cdot}q_0$};
    \node[state,right of=p11,ellipse]                               (p31) {$s_3{\cdot}q_1$};
    \node[state,right of=p31,ellipse,fill=green!20!white,accepting] (p40) {$s_4{\cdot}q_0$};
    \node[state,below of=p00,ellipse]                               (p01) {$s_0{\cdot}q_1$};
    \node[state,right of=p01,ellipse]                               (p10) {$s_1{\cdot}q_0$};
    \node[state,below of=p10,ellipse]                               (p21) {$s_2{\cdot}q_1$};
    \node[state,right of=p10,ellipse,fill=green!20!white,accepting] (p30) {$s_3{\cdot}q_0$};
    \node[state,right of=p30,ellipse]                               (p41) {$s_4{\cdot}q_1$};
    \node[state,right of=p40,ellipse,fill=red!20!white]             (p51) {$s_5{\cdot}q_1$};
    \node[state,right of=p41,ellipse,fill=red!20!white]             (p50) {$s_5{\cdot}q_0$};
   %
    \draw (p00) edge                                          node {0}   (p11)
                edge[bend left=20]                            node {1}   (p01)
          (p01) edge                                          node {0}   (p10)
                edge[bend left=20]                            node {1}   (p00)
          (p10) edge[bend left=20]                            node {0}   (p11)
                edge[left]                                    node {1}   (p21)
          (p11) edge[bend left=20]                            node {0}   (p10)
                edge                                          node {1}   (p20)
          (p20) edge[bend left=40]                            node {0}   (p31)
                edge[out=200,in=150,looseness=2.4,above left] node {1}   (p01)
          (p21) edge[bend right=40,below right]               node {0}   (p30)
                edge[out=160,in=210,looseness=2.4]            node {1}   (p00)
          (p30) edge[bend left=20]                            node {0}   (p31)
                edge[bend left=20]                            node {1}   (p41)
          (p31) edge[bend left=20]                            node {0}   (p30)
                edge[bend left=20]                            node {1}   (p40)
          (p40) edge[bend left=20]                            node {1}   (p31)
                edge                                          node {0}   (p51)
          (p41) edge[bend left=20]                            node {1}   (p30)
                edge                                          node {0}   (p50)
          (p50) edge[bend left=20]                            node {0,1} (p51)
          (p51) edge[bend left=20]                            node {0,1} (p50);
   \end{tikzpicture}
\end{center}
%- - - - - - - - - - - - - - - - - - - - - - - - - - - - - - - - 
\end{itemize}
%
%\noindent\textbf{Obs.:} esta observação e os comandos de cor (``\verb|\textcolor{}{}|'') podem ser removidos da versão a ser submetida na plataforma Turing.
\end{document}
