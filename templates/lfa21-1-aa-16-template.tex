\documentclass[12pt]{article}

% -------------------------------------------------- Dados do discente
% Insira os seus dados e do exercício escolhido:
\def\discente{Fulano de tal}
\def\matricula{20010101}
\def\aa{16}
\def\myling{{99}} % Informe o número da linguagem selecionada.

% -------------------------------------------------- Babel & Geometry
\usepackage[brazil]{babel}
\usepackage[T1]{fontenc}
\usepackage[utf8]{inputenc}
\usepackage[a4paper,top=.5cm,bottom=1cm,left=1cm,right=1cm,includeheadfoot,headheight=52.3318pt,footskip=50pt]{geometry}
%
\usepackage{xcolor}
\usepackage{enumitem}
%
\usepackage{tikz}
\usetikzlibrary{arrows}
\usetikzlibrary{positioning}
%
\usepackage{amsmath,amssymb,amsthm,mathtools}
%
\usepackage{tikz}
\usetikzlibrary{automata,arrows,positioning,shapes}
\tikzset{
 node distance=3cm,
 initial text={$P_{\myling}$},
 double distance=1pt,
 every state/.style={semithick,fill=blue!20!white,minimum size=20pt,inner sep=0pt},
 every edge/.style={draw,->,>=stealth,auto,semithick,font=\ttfamily\small}
}
%
\newcommand{\deriv}[1]{\stackrel{\scriptscriptstyle #1}{\Longrightarrow}}
%

% -------------------------------------------------- Header & Foot
\usepackage{lastpage}
\usepackage{fancyhdr}
\pagestyle{fancy}
\fancyhf{}
\renewcommand{\headrulewidth}{0pt}

\rfoot{\rule{\textwidth}{1.pt}\\\thepage\ de \pageref{LastPage}}

\chead{
 \footnotesize\textbf{Universidade Federal de Goiás -- UFG}\hfill \textsc{Linguagens Formais e Autômatos -- 2021/1}\\
 \footnotesize\textbf{Instituto de Informática -- INF\hfill Prof. Humberto J. Longo} -- \scriptsize\texttt{longo@inf.ufg.br}\\
  \rule{\textwidth}{1.pt}
}

% --------------------------------------------------
\begin{document}
\subsection*{Atividade AA-\aa}
 \paragraph{\matricula -- \discente}
 \begin{itemize}
% 
  \item Nesta tarefa deve ser proposto um PDA que reconheça as cadeias da linguagem $\mathcal{L}$ selecionada na atividade AA-10.
%- - - - - - - - - - - - - - - - - - - - - - - - - - - - - - - - 
  \item \textcolor{red}{$\mathcal{L}_{\myling} = \{a^nb^{2m+1}c^{2m+1}a^{2n} \mid n,m \geqslant 0\}$.}
%- - - - - - - - - - - - - - - - - - - - - - - - - - - - - - - - 
  \item  \textcolor{red}{PDA $P_{\myling}$ que reconhece as cadeias da linguagem $\mathcal{L}_{\myling}$:}
  \end{itemize}
%- - - - - - - - - - - - - - - - - - - - - - - - - - - - - - - -
\begin{center}
\begin{tikzpicture}[
  accept/.style={accepting,fill=green!20!white},
  reject/.style={accepting,fill=red!20!white},
  slopea/.style={sloped,anchor=center,above},
  slopeb/.style={sloped,anchor=center,below},
  transform shape,
%  scale=0.95
 ]
  \def\e{\varepsilon}
  \node[initial,state]           (s0) {$s_0$};
  \node[state,right of=s0]              (s1) {$s_1$};
  \node[state,below of=s1]              (s2) {$s_2$};
  \node[state,right of=s1]              (s3) {$s_3$};
  \node[state,below of=s3]              (s4) {$s_4$};
  \node[state,right of=s3]              (s5) {$s_5$};
  \node[state,below of=s5]              (s6) {$s_6$};
  \node[state,right of=s5]              (s7) {$s_7$};
  \node[state,accept,right of=s7]       (sa) {$s_a$};
  \node[state,reject,below of=s4]       (sr) {$s_r$};

  \path (s0) edge node {$\e,\e\to \$$}                (s1)
        (s1) edge [bend right=20,left]
                  node [slopeb] {$a,\e\to a$}         (s2)
             edge node {$b,\e\to b$}                  (s3)
             edge [out=50,in=100,looseness=0.6]
                  node [slopea,pos=.3] {$\e,\$\to\e$} (sa)
             edge [out=220,in=150,looseness=1.2]
                  node [slopeb,pos=.8] {$c,\e\to\e$}  (sr)
        (s2) edge [bend right=20,right]
                  node [slopeb] {$\e,\e\to a$}        (s1)
        (s3) edge [bend right=20,left]
                  node [slopeb] {$b,\e\to b$}         (s4)
             edge node {$c,b\to \e$}                  (s5)
             edge [out=210,in=120,looseness=0.9]
                  node [slopeb,pos=.7] {$a,\e\to\e$}  (sr)
        (s4) edge [bend right=20,left]
                  node [slopeb] {$b,\e\to b$}         (s3)
             edge node[slopea,align=left] {$a,\e\to\e$\\[-2pt] $c,\e\to\e$} (sr)
        (s5) edge [bend right=20,left]
                  node [slopeb] {$c,b\to \e$}         (s6)
             edge node {$a,a\to \e$}                  (s7)
             edge [out=40,in=140,looseness=1.1]
                  node [slopea,pos=.2] {$\e,\$\to\e$} (sa)
             edge [out=335,in=0,looseness=1.1]
                  node [slopeb,pos=.4] {$b,\e\to\e$}  (sr)
        (s6) edge [bend right=20,left]
                  node [slopeb] {$c,b\to\e$}          (s5)
             edge [bend left=20]
                  node[slopea,pos=.3,align=left] {$a,\e\to\e$\\[-2pt] $b,\e\to \e$}  (sr)
        (s7) edge [loop above]
                  node {$a,a\to\e$}                   (s7)
             edge node {$\e,\$\to\e$}                 (sa)
             edge [out=260,in=340,looseness=0.9]
                  node[slopeb,align=left] {$b,\e\to\e$\\[-2pt] $c,\e\to\e$} (sr);
\end{tikzpicture}
\end{center}
%- - - - - - - - - - - - - - - - - - - - - - - - - - - - - - - -
%
\noindent\textbf{Obs.:} Esta observação e os comandos de cor (``\verb|\textcolor{}{}|'') podem ser removidos da versão a ser submetida na plataforma Turing.
\end{document}
