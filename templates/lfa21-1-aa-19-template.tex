\documentclass[12pt]{article}

% -------------------------------------------------- Dados do discente
% Insira os seus dados e do exercício escolhido:
\def\discente{Fulano de tal}
\def\matricula{20010101}
\def\aa{19}
\def\myling{{99}} % Informe o número da linguagem selecionada.

% -------------------------------------------------- Babel & Geometry
\usepackage[brazil]{babel}
\usepackage[T1]{fontenc}
\usepackage[utf8]{inputenc}
\usepackage[a4paper,top=.5cm,bottom=1cm,left=1cm,right=1cm,includeheadfoot,headheight=52.3318pt,footskip=50pt]{geometry}
%
\usepackage{xcolor}
\usepackage{enumitem}
\usepackage{marvosym} % \Smiley
%
\usepackage{tikz}
\usetikzlibrary{arrows}
\usetikzlibrary{positioning}
%
\usepackage{amsmath,amssymb,amsthm,mathtools}
%
\usepackage{tikz}
\usetikzlibrary{automata,arrows,positioning,shapes}
\tikzset{
 node distance=3cm,
 initial text={$P_{\myling}$},
 double distance=1pt,
 every state/.style={semithick,fill=blue!20!white,minimum size=20pt,inner sep=0pt},
 every edge/.style={draw,->,>=stealth,auto,semithick,font=\ttfamily\small}
}
%
\newcommand{\deriv}[1]{\stackrel{\scriptscriptstyle #1}{\Longrightarrow}}
%

% -------------------------------------------------- Header & Foot
\usepackage{lastpage}
\usepackage{fancyhdr}
\pagestyle{fancy}
\fancyhf{}
\renewcommand{\headrulewidth}{0pt}

\rfoot{\rule{\textwidth}{1.pt}\\\thepage\ de \pageref{LastPage}}

\chead{
 \footnotesize\textbf{Universidade Federal de Goiás -- UFG}\hfill \textsc{Linguagens Formais e Autômatos -- 2021/1}\\
 \footnotesize\textbf{Instituto de Informática -- INF\hfill Prof. Humberto J. Longo} -- \scriptsize\texttt{longo@inf.ufg.br}\\
  \rule{\textwidth}{1.pt}
}

\newcommand\tvs[1][.8em]{%
  \makebox[#1]{%
    \kern.07em
    \vrule height.5ex
    \hrulefill
    \vrule height.5ex
    \kern.07em
  }
}

% --------------------------------------------------
\begin{document}
\subsection*{Atividade AA-\aa}
 \paragraph{\matricula -- \discente}
 \begin{itemize}
% 
  \item \textcolor{blue}{Nesta tarefa deve-se ($i$) especificar a 7-tupla que define uma máquina de Turing que decide a linguagem escolhida na AA-18 e ($ii$) construir o respectivo diagrama de estados.}
%- - - - - - - - - - - - - - - - - - - - - - - - - - - - - - - - 
  \item \textcolor{red}{$\mathcal{L}_{\myling} = \{0^n1^n2^n \mid n \geqslant 0\}$.}
%- - - - - - - - - - - - - - - - - - - - - - - - - - - - - - - - 
  \item  \textcolor{red}{Máquina de Turing $M_{\myling}$ que reconhece as cadeias da linguagem $\mathcal{L}_{\myling}$:
  \begin{align*}
  M_{\myling} &= (S,\Sigma,\Gamma,\delta,s_0,s_a,s_r),\shortintertext{com}
            S &= \{s_o,s_1,s_2,s_3,s_4,s_5,s_6,s_a,s_r\},\\
       \Sigma &= \{0,1,2\},\\
       \Gamma &= \{0,1,2,X,Y,\tvs\}.
  \end{align*}}
  \item  \textcolor{red}{Diagrama de estados da Máquina de Turing $M_{\myling}$:}
  \end{itemize}
%- - - - - - - - - - - - - - - - - - - - - - - - - - - - - - - -
\begin{center}
\begin{tikzpicture}[
  accept/.style={accepting,fill=green!20!white},
  reject/.style={accepting,fill=red!20!white},
  slopea/.style={sloped,anchor=center,above},
  slopeb/.style={sloped,anchor=center,below},
%  transform shape,
%  scale=1.2
]
  \node[initial below,state]            (s0) {$s_0$};
  \node[state,reject,above right of=s0] (sr) {$s_r$};
  \node[state,above left of=s0]         (s1) {$s_1$};
  \node[state,below left of=s1]         (s2) {$s_2$};
  \node[state,below right of=sr]        (s3) {$s_3$};
  \node[state,above right of=s3]        (s4) {$s_4$};
  \node[state,below right of=s4]        (s5) {$s_5$};
  \node[state,right of=s5]              (s6) {$s_6$};
  \node[state,accept,below right of=s0] (sa) {$s_a$};
%
  \path (s0) edge [slopea]
                  node {$0\to\tvs,D$}   (s1)
             edge node {$X\to\tvs,D$}   (s3)
             edge [slopea]
                  node {$\tvs\to E$}     (sa)
             edge [slopea]
                  node {$1,2\to D$}      (sr)
        (s1) edge [loop above,pos=.3]
                  node {$0,X\to D$}      (s1)
             edge [slopea]
                  node {$1\to X, E$}     (s2)
             edge [slopea]
                  node {$\tvs\to D$}     (sr)
        (s2) edge [loop below]
                  node {$0,X\to E$}      (s2)
             edge node {$\tvs\to D$}     (s0)
        (s3) edge [loop below]
                  node {$X,Y\to D$}      (s3)
             edge [slopea]
                  node {$2\to Y,E$}      (s4)
             edge [slopea]
                  node {$0,1\to D$}      (sr)
        (s4) edge [loop above,pos=.3]
                  node {$X,Y\to E$}      (s4)
             edge [slopea]
                  node {$\tvs\to D$}     (s5)
        (s5) edge [above]
                  node {$X\to \tvs,D$}   (s3)
             edge node {$Y\to D$}        (s6)
        (s6) edge [out=220,in=350,pos=.7,slopea]
                  node {$\tvs\to E$}     (sa)
             edge [out=110,in=40,pos=.3,slopea]
                  node {$0,1,2\to D$}    (sr);
\end{tikzpicture}
\end{center}
%- - - - - - - - - - - - - - - - - - - - - - - - - - - - - - - -
%
\noindent\textbf{Obs.:} Esta observação e os comandos de cor (``\verb|\textcolor{}{}|'') podem ser removidos da versão a ser submetida na plataforma Turing.
\end{document}
