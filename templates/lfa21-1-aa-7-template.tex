\documentclass[12pt]{article}

% -------------------------------------------------- Dados do discente
% Insira os seus dados e do exercício escolhido:
\def\discente{Fulano de tal}
\def\matricula{20010101}
\def\aa{7}
\def\myling{{99}} % Informe o número da linguagem selecionada.

% -------------------------------------------------- Babel & Geometry
\usepackage[brazil]{babel}
\usepackage[T1]{fontenc}
\usepackage[utf8]{inputenc}
\usepackage[a4paper,top=.5cm,bottom=1cm,left=1cm,right=1cm,includeheadfoot,headheight=52.3318pt,footskip=50pt]{geometry}

% -------------------------------------------------- TikZ
\usepackage{tikz}
\usetikzlibrary{automata,arrows,positioning}
\tikzset{
 node distance=2.5cm,
 initial text={$M_\myling$},
 double distance=1pt,
 every state/.style={semithick,fill=blue!20!white,minimum size=20pt,inner sep=0pt},
 every edge/.style={draw,->,>=stealth,auto,semithick,font=\ttfamily\small}
}

% -------------------------------------------------- Header & Foot
\usepackage{lastpage}
\usepackage{fancyhdr}
\pagestyle{fancy}
\fancyhf{}
\renewcommand{\headrulewidth}{0pt}

\rfoot{\rule{\textwidth}{1.pt}\\\thepage\ de \pageref{LastPage}}

\chead{
 \footnotesize\textbf{Universidade Federal de Goiás -- UFG}\hfill \textsc{Linguagens Formais e Autômatos -- 2021/1}\\
 \footnotesize\textbf{Instituto de Informática -- INF\hfill Prof. Humberto J. Longo} -- \scriptsize\texttt{longo@inf.ufg.br}\\
  \rule{\textwidth}{1.pt}
}

% --------------------------------------------------
\begin{document}
\subsection*{Atividade AA-\aa}
 \paragraph{\matricula -- \discente}
%
 \begin{itemize}
% - - - - - - - - - - - - - - - - - - - - - - - - - - - - - - - - - - - -
  \item A partir dos autômatos propostos nas atividades avaliativas AA-4 (DFA mínimo) e AA-6 (NFA), construa gramáticas que gerem as cadeias da linguagem que esses autômatos reconhecem. Caso os autômatos propostos na AA-4 e/ou AA-6 tenham sido avaliados como incorretos, devem ser usadas versões corrigidas dos mesmos.  A gramática obtida a partir do DFA \textbf{deve ser regular} e a gramática resultante do NFA não necessariamente será regular! 
% - - - - - - - - - - - - - - - - - - - - - - - - - - - - - - - - - - - -
  \item \textcolor{blue}{$\mathcal{L}_\myling = \{w\mid w=0u1$ ou $w=1u0$, com $u\in\Sigma^*\}$.}
% - - - - - - - - - - - - - - - - - - - - - - - - - - - - - - - - - - - -
  \item  \textcolor{red}{Autômato finito determinístico que reconhece as cadeias $\mathcal{L}_\myling$:}
\begin{center}
 \begin{tikzpicture}
%  [node distance=2.cm,
%    transform shape,
%    scale=1.3
%  ]
    \node[state,initial]                                   (s0) {$s_0$};
    \node[state,above right=1cm of s0]                     (s1) {$s_1$};
    \node[state,below right=1cm of s0]                     (s2) {$s_2$};
    \node[state,right of=s1,accepting,fill=green!20!white] (s3) {$s_3$};
    \node[state,right of=s2,accepting,fill=green!20!white] (s4) {$s_4$};
  %
    \draw (s0) edge [bend left=45]              node {$0$} (s1)
               edge [bend right=45, below left] node {$1$} (s2)
          (s1) edge [loop above]                node {$0$} (s1)
               edge [bend left=20]              node {$1$} (s3)
          (s2) edge [bend left=20]              node {$0$} (s4)
               edge [loop above]                node {$1$} (s1)
          (s3) edge [bend left=20]              node {$0$} (s1)
               edge [loop above]                node {$1$} (s3)
          (s4) edge [loop above]                node {$0$} (s4)
               edge [bend left=20]              node {$1$} (s2);
   \end{tikzpicture}
\end{center}
% - - - - - - - - - - - - - - - - - - - - - - - - - - - - - - - - - - - -
  \item  \textcolor{red}{NFA $M_\myling$ com os estados renomeados:}
\begin{center}
 \begin{tikzpicture}
%  [node distance=2.cm,
%    transform shape,
%    scale=1.3
%  ]
    \node[state,initial]                                   (s0) {$S$};
    \node[state,above right=1cm of s0]                     (s1) {$A$};
    \node[state,below right=1cm of s0]                     (s2) {$C$};
    \node[state,right of=s1,accepting,fill=green!20!white] (s3) {$B$};
    \node[state,right of=s2,accepting,fill=green!20!white] (s4) {$D$};
  %
    \draw (s0) edge [bend left=45]              node {$0$} (s1)
               edge [bend right=45, below left] node {$1$} (s2)
          (s1) edge [loop above]                node {$0$} (s1)
               edge [bend left=20]              node {$1$} (s3)
          (s2) edge [bend left=20]              node {$0$} (s4)
               edge [loop above]                node {$1$} (s2)
          (s3) edge [bend left=20]              node {$0$} (s1)
               edge [loop above]                node {$1$} (s3)
          (s4) edge [loop above]                node {$0$} (s4)
               edge [bend left=20]              node {$1$} (s2);
   \end{tikzpicture}
\end{center}
% - - - - - - - - - - - - - - - - - - - - - - - - - - - - - - - - - - - -
  \item  \textcolor{red}{Gramática $G_1$ que gera as cadeias da linguagem $\mathcal{L}_{\myling}$:\\ $G_1=(V,\Sigma,P,S)=(\{A,B,C,D,S\},\{0,1\},P,S)$, com
  $$
   P =
   \left\{\begin{array}{l}
    S \to 0A \mid 1C,\\
    A \to 0A \mid 1B,\\
    B \to 0A \mid 1B \mid \varepsilon,\\
    C \to 0D \mid 1C,\\
    D \to 0D \mid 1C \mid \varepsilon
   \end{array}\right\}.
  $$
  }
% - - - - - - - - - - - - - - - - - - - - - - - - - - - - - - - - - - - -
  \item  \textcolor{red}{Autômato finito não determinístico que reconhece as cadeias $\mathcal{L}_\myling$:}
\begin{center}
 \begin{tikzpicture}
  [
   initial text={$N_\myling$},
%    node distance=2.cm,
%    transform shape,
%    scale=1.3
  ]
    \node[state,initial]                                             (s0) {$s_0$};
    \node[state,above right=1cm of s0]                               (s1) {$s_1$};
    \node[state,right of=s1]                                         (s2) {$s_2$};
    \node[state,right of=s2]                                         (s3) {$s_3$};
    \node[state,right of=s3]                                         (s4) {$s_4$};
    \node[state,below right=1cm of s0]                               (s5) {$s_5$};
    \node[state,right of=s5]                                         (s6) {$s_6$};
    \node[state,right of=s6]                                         (s7) {$s_7$};
    \node[state,right of=s7]                                         (s8) {$s_8$};
    \node[state,above right=1cm of s8,accepting,fill=green!20!white] (s9) {$s_9$};
  %
    \draw (s0) edge [bend left=45]              node {$\varepsilon$}     (s1)
               edge [bend right=45, below left] node {$\varepsilon$}     (s5)
          (s1) edge                             node {$0$}               (s2)
          (s2) edge [bend left=20]              node {$\varepsilon,0,1$} (s3)
          (s3) edge [bend left=20]              node {$\varepsilon$}     (s2)
               edge                             node {$1$}               (s4)
          (s4) edge [bend left=45]              node {$\varepsilon$}     (s9)
          (s5) edge                             node {$1$}               (s6)
          (s6) edge [bend left=20]              node {$\varepsilon,0,1$} (s7)
          (s7) edge [bend left=20]              node {$\varepsilon$}     (s6)
               edge                             node {$0$}               (s8)
          (s8) edge [bend right=45,below right] node {$\varepsilon$}     (s9);
   \end{tikzpicture}
\end{center}
% - - - - - - - - - - - - - - - - - - - - - - - - - - - - - - - - - - - -
  \item  \textcolor{red}{NFA $N_\myling$ com os estados renomeados:}
\begin{center}
 \begin{tikzpicture}
  [
   initial text={$N_\myling$},
%    node distance=2.cm,
%    transform shape,
%    scale=1.3
  ]
    \node[state,initial]                                             (s0) {$S$};
    \node[state,above right=1cm of s0]                               (s1) {$A$};
    \node[state,right of=s1]                                         (s2) {$B$};
    \node[state,right of=s2]                                         (s3) {$C$};
    \node[state,right of=s3]                                         (s4) {$D$};
    \node[state,below right=1cm of s0]                               (s5) {$E$};
    \node[state,right of=s5]                                         (s6) {$F$};
    \node[state,right of=s6]                                         (s7) {$G$};
    \node[state,right of=s7]                                         (s8) {$H$};
    \node[state,above right=1cm of s8,accepting,fill=green!20!white] (s9) {$I$};
  %
    \draw (s0) edge [bend left=45]              node {$\varepsilon$}     (s1)
               edge [bend right=45, below left] node {$\varepsilon$}     (s5)
          (s1) edge                             node {$0$}               (s2)
          (s2) edge [bend left=20]              node {$\varepsilon,0,1$} (s3)
          (s3) edge [bend left=20]              node {$\varepsilon$}     (s2)
               edge                             node {$1$}               (s4)
          (s4) edge [bend left=45]              node {$\varepsilon$}     (s9)
          (s5) edge                             node {$1$}               (s6)
          (s6) edge [bend left=20]              node {$\varepsilon,0,1$} (s7)
          (s7) edge [bend left=20]              node {$\varepsilon$}     (s6)
               edge                             node {$0$}               (s8)
          (s8) edge [bend right=45,below right] node {$\varepsilon$}     (s9);
   \end{tikzpicture}
\end{center}
% - - - - - - - - - - - - - - - - - - - - - - - - - - - - - - - - - - - -
  \item  \textcolor{red}{Gramática $G_2$ que gera as cadeias da linguagem $\mathcal{L}_{\myling}$:\\ $G_2=(V,\Sigma,P,S)=(\{A,B,C,D,E,F,G,H,I,S\},\{0,1\},P,S)$, com
  $$
   P =
   \left\{\begin{array}{l}
    S \to A \mid E,\\
    A \to 0B,\\
    B \to C \mid 0C \mid 1C\\
    C \to B \mid 1D,\\
    D \to I\\
    E \to 1F,\\
    F \to G \mid 0G \mid 1G\\
    G \to F \mid 0H,\\
    H \to I\\
    I \to \varepsilon
   \end{array}\right\}.
  $$
  }
% - - - - - - - - - - - - - - - - - - - - - - - - - - - - - - - - - - - -
 \end{itemize}
% - - - - - - - - - - - - - - - - - - - - - - - - - - - - - - - - - - - -
\noindent\textbf{Obs.:} esta observação e os comandos de cor (``\verb|\textcolor{}{}|'') podem ser removidos da versão a ser submetida na plataforma Turing.
\end{document}
