\documentclass[12pt]{article}

% -------------------------------------------------- Dados do discente
% Insira os seus dados e do exercício escolhido:
\def\discente{Fulano de tal}
\def\matricula{20010101}
\def\aa{14}
\def\myling{{99}} % Informe o número da linguagem selecionada.

% -------------------------------------------------- Babel & Geometry
\usepackage[brazil]{babel}
\usepackage[T1]{fontenc}
\usepackage[utf8]{inputenc}
\usepackage[a4paper,top=.5cm,bottom=1cm,left=1cm,right=1cm,includeheadfoot,headheight=52.3318pt,footskip=50pt]{geometry}
%
\usepackage{xcolor}
\usepackage{enumitem}
%
\usepackage{tikz}
\usetikzlibrary{arrows}
\usetikzlibrary{positioning}
%
\usepackage{amsmath,amssymb,amsthm,mathtools}
%
\newcommand{\deriv}[1]{\stackrel{\scriptscriptstyle #1}{\Longrightarrow}}
%

% -------------------------------------------------- Header & Foot
\usepackage{lastpage}
\usepackage{fancyhdr}
\pagestyle{fancy}
\fancyhf{}
\renewcommand{\headrulewidth}{0pt}

\rfoot{\rule{\textwidth}{1.pt}\\\thepage\ de \pageref{LastPage}}

\chead{
 \footnotesize\textbf{Universidade Federal de Goiás -- UFG}\hfill \textsc{Linguagens Formais e Autômatos -- 2021/1}\\
 \footnotesize\textbf{Instituto de Informática -- INF\hfill Prof. Humberto J. Longo} -- \scriptsize\texttt{longo@inf.ufg.br}\\
  \rule{\textwidth}{1.pt}
}

% --------------------------------------------------
\begin{document}
\subsection*{Atividade AA-\aa}
 \paragraph{\matricula -- \discente}
%
 \begin{itemize}
%- - - - - - - - - - - - - - - - - - - - - - - - - - - - - - - - 
  \item Suponha que $G_0$ é a gramática proposta na atividade avaliativa AA-12, ou seja, $G_0$ é igual à gramática proposta na atividade avaliativa AA-11 ou é uma versão corrigida da mesma. Obtenha as gramáticas $G_i$, $i=1,\dots,5$, especificadas em cada um dos passos a seguir (eventualmente pode ser que $G_i=G_{i-1}$, para algum $i\in\{1,\dots,5\}$):
  \begin{enumerate}[topsep=-1pt,itemsep=0pt]
   \item elimine recursão na variável inicial de $G_0$ e obtenha $G_1$;
   \item elimine as $\varepsilon$-regras de $G_1$ e obtenha $G_2$;
   \item elimine derivações simples de $G_2$ e obtenha $G_3$;
   \item elimine recursões à esquerda de $G_3$ e obtenha $G_4$; e
   \item elimine símbolos inúteis de $G_4$ e obtenha $G_5$.
  \end{enumerate}
  
%- - - - - - - - - - - - - - - - - - - - - - - - - - - - - - - - 
  \item \textcolor{red}{$\mathcal{L}_{\myling} = \{a^nb^{2m+1}c^{2m+1}a^{2n} \mid n,m \geqslant 0\}$.}
%- - - - - - - - - - - - - - - - - - - - - - - - - - - - - - - - 
  \item  \textcolor{red}{Gramática $G_{\myling}$ que gera as cadeias da linguagem $\mathcal{L}_{\myling}$:\\ $G_{\myling}=(V,\Sigma,P,S)=(\{A,B,S\},\{a,b,c\},P,S)$, com
  $$
   P =
   \left\{\begin{array}{l}
    S\to aSaa\mid A\\
    A\to bbAcc\mid B\\
    B\to bc
   \end{array}\right\}.
  $$
  }
\end{itemize}
%- - - - - - - - - - - - - - - - - - - - - - - - - - - - - - - -
\textcolor{blue}{
\begin{enumerate}
  \item Eliminar recursão na variável inicial de $G_{\myling}$ e obter $G_{\myling}^1$:\\
  $G_{\myling}^1=(V,\Sigma,P,S)=(\{A,B,S_0\},\{a,b,c\},P,S)$, com
    $$
     P =
     \left\{\begin{array}{l}
      S_0\to S\\
      S\to aSaa\mid A\\
      A\to bbAcc\mid B\\
      B\to bc
     \end{array}\right\}.
    $$
%- - - - - - - - - - - - - - - - - - - - - - - - - - - - - - - -
  \item Eliminar $\varepsilon$-regras de $G_{\myling}^1$ e obter $G_{\myling}^2$:\\
  $G_{\myling}^2=(V,\Sigma,P,S_0)=(\{A,B,S,S_0\},\{a,b,c\},P,S_0)$, com
    $$
     P =
     \left\{\begin{array}{l}
      S_0\to S\\
      S\to aSaa\mid A\\
      A\to bbAcc\mid B\\
      B\to bc
     \end{array}\right\}.
    $$
%- - - - - - - - - - - - - - - - - - - - - - - - - - - - - - - -
  \item Eliminar derivações simples de $G_{\myling}^2$ e obter $G_{\myling}^3$:\\
  $G_{\myling}^3=(V,\Sigma,P,S_0)=(\{A,B,S,S_0\},\{a,b,c\},P,S_0)$, com
    $$
     P =
     \left\{\begin{array}{l}
      S_0\to aSaa\mid bbAcc\mid bc\\
      S\to aSaa\mid bbAcc\mid bc\\
      A\to bbAcc\mid bc\\
      B\to bc
     \end{array}\right\}.
    $$
%- - - - - - - - - - - - - - - - - - - - - - - - - - - - - - - -
  \item Eliminar recursões à esquerda de $G_{\myling}^3$ e obter $G_{\myling}^4$:\\
  $G_{\myling}^4=(V,\Sigma,P,S_0)=(\{A,B,S,S_0\},\{a,b,c\},P,S_0)$, com
    $$
     P =
     \left\{\begin{array}{l}
      S_0\to aSaa\mid bbAcc\mid bc\\
      S\to aSaa\mid bbAcc\mid bc\\
      A\to bbAcc\mid bc\\
      B\to bc
     \end{array}\right\}.
    $$
%- - - - - - - - - - - - - - - - - - - - - - - - - - - - - - - -
  \item Eliminar símbolos inúteis de $G_{\myling}^4$ e obter $G_{\myling}^5$:\\
  $G_{\myling}^5=(V,\Sigma,P,S_0)=(\{A,S,S_0\},\{a,b,c\},P,S_0)$, com
    $$
     P =
     \left\{\begin{array}{l}
      S_0\to aSaa\mid bbAcc\mid bc\\
      S\to aSaa\mid bbAcc\mid bc\\
      A\to bbAcc\mid bc
     \end{array}\right\}.
    $$
\end{enumerate}
}
%- - - - - - - - - - - - - - - - - - - - - - - - - - - - - - - -
%
\noindent\textbf{Obs.:} Esta observação e os comandos de cor (``\verb|\textcolor{}{}|'') podem ser removidos da versão a ser submetida na plataforma Turing.
\end{document}
