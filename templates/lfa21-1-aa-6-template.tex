\documentclass[12pt]{article}

% --------------------------------------------------
% Insira os seus dados e do exercício escolhido:
\def\discente{Fulano de tal}
\def\matricula{20010101}
\def\aa{6}
\def\myling{{99}} % Informe o número da linguagem selecionada.
% --------------------------------------------------

\usepackage[brazil]{babel}
\usepackage[T1]{fontenc}
\usepackage[utf8]{inputenc}
\usepackage[a4paper,top=1.0cm,bottom=1.0cm,left=2.0cm,right=1.5cm,nohead,nofoot]{geometry}

\usepackage{amssymb}


\usepackage{tikz}
\usetikzlibrary{automata,arrows,positioning,shapes}
\tikzset{
 node distance=2.0cm,
 initial text={$N_{\myling}$},
 double distance=1pt,
 every state/.style={semithick,fill=blue!20!white,minimum size=20pt,inner sep=0pt},
 every edge/.style={draw,->,>=stealth,auto,semithick,font=\ttfamily\small},
 accept/.style={accepting,fill=green!20!white},
 reject/.style={accepting,fill=red!20!white},
}


\begin{document}
\pagestyle{empty}
\subsection*{Atividade AA-\aa}
 \paragraph{\matricula -- \discente}
%
 \begin{itemize}
%- - - - - - - - - - - - - - - - - - - - - - - - - - - - - - - 
  \item [1)] A partir da expressão regular $\mathcal{R}$ proposta na AA-3 construa um \textbf{autômato finito não-determinístico} (NFA ou NFA-$\varepsilon$) que reconheça $\mathcal{L}(\mathcal{R})$. \textbf{Obs.:} Caso a expressão regular proposta na AA-3 tenha sido avaliada como incorreta, use uma versão corrigida da mesma.
  \item [2)] Converta o NFA/NFA-$\varepsilon$ obtido para um autômato finito determinístico (apresente os principais elementos dessa conversão).
%- - - - - - - - - - - - - - - - - - - - - - - - - - - - - - - - 
  \item \textcolor{blue}{$\mathcal{R}_{\myling} = 0^*\cup 0(10)^*$.}
%- - - - - - - - - - - - - - - - - - - - - - - - - - - - - - - - 
  \item  \textcolor{red}{Autômato finito não determinístico que reconhece as cadeias da linguagem $\mathcal{L}(\mathcal{R}_{\myling})$:}\\
   \begin{tikzpicture}[
%    transform shape,
%    scale=1.2
   ]
    \node[state,initial]                  (s0) {$s_0$};
    \node[state,above right of=s0]        (s1) {$s_1$};
    \node[state,below right of=s0,accept] (s2) {$s_2$};
    \node[state,right of=s1,accept]       (s3) {$s_3$};

    \draw (s0) edge [bend left=45]             node {$\varepsilon$} (s1)
               edge [bend right=45,below left] node {$\varepsilon$} (s2)
          (s1) edge [bend left=20]             node {$0$}           (s3)
          (s2) edge [loop above]               node {$0$}           (s2)
          (s3) edge [bend left=20]             node {$1$}           (s1);
  \end{tikzpicture}
%- - - - - - - - - - - - - - - - - - - - - - - - - - - - - - - - 
 \item \textcolor{red}{Fecho-$\varepsilon$ do estado $s_0$:} $$\mathcal{F}_\varepsilon(s_0) = \{s_0,s_1,s_2\}.$$
%- - - - - - - - - - - - - - - - - - - - - - - - - - - - - - - - 
 \item \textcolor{red}{Tabela da função $\tau$ de transições:}
 $$
  \begin{array}{c|cc}
   \tau & 0           & 1\\
   \hline
   s_0  & \{s_2,s_3\} & \varnothing\\
   s_1  & \{s_3\}     & \varnothing\\
   s_2  & \{s_2\}     & \varnothing\\
   s_3  & \varnothing & \{s_1\}\\
  \end{array}
 $$
%- - - - - - - - - - - - - - - - - - - - - - - - - - - - - - - - 
  \item \textcolor{red}{Autômato finito determinístico que reconhece as cadeias da linguagem $\mathcal{L}(\mathcal{R}_{\myling})$:}\\
   \begin{tikzpicture}[
     node distance=3.0cm,
     initial text={$M_{\myling}$}
%    transform shape,
%    scale=1.2
   ]
    \node[state,initial,accept,ellipse]                  (s012) {$\{s_0,s_1,s_2\}$};
    \node[state,right of=s012,accept,ellipse]            (s23)  {$\{s_2,s_3\}$};
    \node[state,right of=s23,ellipse]                    (s1)   {$\{s_1\}$};
    \node[state,right of=s1,accept,ellipse]              (s3)   {$\{s_3\}$};
    \node[state,below of=s23,accept,ellipse,yshift=1cm]  (s2)   {$\{s_2\}$};
    \node[state,below of=s2,reject,ellipse,yshift=1cm]   (sr)   {$\varnothing$};

    \draw (s012) edge                            node {$0$}   (s23)
                 edge [bend right=45,below left] node {$1$}   (sr)
          (s23)  edge                node {$0$}   (s2)
                 edge                node {$1$}   (s1)
          (s1)   edge [bend left=20] node {$0$}   (s3)
                 edge [bend left=20] node {$1$}   (sr)
          (s3)   edge [bend left=40] node {$0$}   (sr)
                 edge [bend left=20] node {$1$}   (s1)
          (s2)   edge [loop left]    node {$0$}   (s2)
                 edge                node {$1$}   (sr)
          (sr)   edge [loop below]   node {$0,1$} (sr);
  \end{tikzpicture}
%- - - - - - - - - - - - - - - - - - - - - - - - - - - - - - - - 
\end{itemize}
%

\noindent\textbf{Obs.:} esta observação e os comandos de cor (``\verb|\textcolor{}{}|'') podem ser removidos da versão a ser submetida na plataforma Turing.
\end{document}
