\documentclass[12pt]{article}

% -------------------------------------------------- Dados do discente
% Insira os seus dados e do exercício escolhido:
\def\discente{Fulano de tal}
\def\matricula{20010101}
\def\aa{13}
\def\myling{{99}} % Informe o número da linguagem selecionada.

% -------------------------------------------------- Babel & Geometry
\usepackage[brazil]{babel}
\usepackage[T1]{fontenc}
\usepackage[utf8]{inputenc}
\usepackage[a4paper,top=.5cm,bottom=1cm,left=1cm,right=1cm,includeheadfoot,headheight=52.3318pt,footskip=50pt]{geometry}
%
\usepackage{xcolor}
\usepackage{enumitem}
%
\usepackage{tikz}
\usetikzlibrary{arrows}
\usetikzlibrary{positioning}
%
\usepackage{amsmath,amssymb,amsthm,mathtools}
%
\newcommand{\deriv}[1]{\stackrel{\scriptscriptstyle #1}{\Longrightarrow}}
%

% -------------------------------------------------- Header & Foot
\usepackage{lastpage}
\usepackage{fancyhdr}
\pagestyle{fancy}
\fancyhf{}
\renewcommand{\headrulewidth}{0pt}

\rfoot{\rule{\textwidth}{1.pt}\\\thepage\ de \pageref{LastPage}}

\chead{
 \footnotesize\textbf{Universidade Federal de Goiás -- UFG}\hfill \textsc{Linguagens Formais e Autômatos -- 2021/1}\\
 \footnotesize\textbf{Instituto de Informática -- INF\hfill Prof. Humberto J. Longo} -- \scriptsize\texttt{longo@inf.ufg.br}\\
  \rule{\textwidth}{1.pt}
}

% --------------------------------------------------
\begin{document}
\subsection*{Atividade AA-\aa}
 \paragraph{\matricula -- \discente}
%
 \begin{itemize}
%- - - - - - - - - - - - - - - - - - - - - - - - - - - - - - - - 
 \item \textcolor{blue}{Suponha que $G$ é a gramática usada na atividade avaliativa AA-12. Note-se que $G$ é igual à gramática proposta na AA-11 ou é uma versão corrigida da mesma. Nesta tarefa verifique se $G$ é ambígua. Em caso afirmativo,
 \begin{enumerate}
  \item mostre duas derivações à esquerda em $G$ para uma mesma cadeia, e
  \item proponha uma gramática equivalente $G'$ que não seja ambígua. 
 \end{enumerate} 
 Caso $G$ não seja ambígua, então:
 \begin{enumerate}
  \item proponha uma gramática equivalente $G''$ que seja ambígua e
  \item mostre duas derivações à esquerda em $G''$ para uma mesma cadeia.
  \end{enumerate}
 }
%- - - - - - - - - - - - - - - - - - - - - - - - - - - - - - - - 
  \item \textcolor{red}{$\mathcal{L}_{\myling} = \{w\in\{(,)\}^*\mid\text{os símbolos `(' e `)' em $w$ estão balanceados} \}$.}
%- - - - - - - - - - - - - - - - - - - - - - - - - - - - - - - - 
  \item  \textcolor{red}{Gramática $G_{\myling}$ que gera as cadeias da linguagem $\mathcal{L}_{\myling}$:\\ 
  $G_{\myling}=(V,\Sigma,P,S)=(\{S\},\{(,)\},P,S)$, com
  $$
   P =
   \left\{\begin{array}{l}
    S \to (S) \mid SS \mid \varepsilon
   \end{array}\right\}.
  $$
  }
%- - - - - - - - - - - - - - - - - - - - - - - - - - - - - - - -
  \item \textcolor{red}{A gramática $G_{\myling}$ é ambígua, pois existem duas derivações à esquerda ( árvores de derivações) distintas para a cadeia ``(())'':
  \begin{align*}
   S &\Rightarrow (S) \Rightarrow (SS) \Rightarrow ((S)S) \Rightarrow ((\varepsilon)S) \Rightarrow (()\varepsilon) \equiv (())\shortintertext{e}
   S &\Rightarrow (S) \Rightarrow (SS) \Rightarrow (\varepsilon S) \Rightarrow ((S)) \Rightarrow  ((\varepsilon)) \equiv (()).
  \end{align*}
  \begin{center}
   \begin{tikzpicture}[
    ->,>=stealth',
    level distance=1.5cm,
    level 2/.style={sibling distance=2.0cm},
    level 3/.style={sibling distance=1cm},
    scale=.7
   ]
    \node (t1) {$S$}
      child { node {(} }
      child { node {$S$}
        child { node {$S$}
          child { node {(} }
          child { node {$S$} 
            child { node {$\varepsilon$}}}
          child { node {)} } }
        child { node {$S$}
          child { node {$\varepsilon$}}}}
      child { node {)} };
%      
    \node[xshift=4cm] (t2) {$S$}
      child { node {(} }
      child { node {$S$}
        child { node {$S$}
          child { node {$\varepsilon$}}}
        child { node {$S$}
          child { node {(} }
          child { node {$S$} 
            child { node {$\varepsilon$}}}
          child { node {)} } }}
      child { node {)} };
   \end{tikzpicture}
  \end{center}
  }
  \item \textcolor{red}{A regra de derivação $S\to \varepsilon$  e a regra recursiva $S\to SS$ da gramática $G_{\myling}$ permitem gerar qualquer quantidade da variável $S$, as quais combinadas com a regra recursiva $S\to (S)$ permitem que diferentes derivações à esquerda sejam construídas para uma mesma cadeia!
  }
  \item \textcolor{red}{A introdução de uma variável $A$ extra elimina uma das recursões diretas na variável $S$ e remove a ambiguidade da gramáticas, ou seja, $S\to \varepsilon \mid AS$ e $A\to (S)$. Note-se que essas regras podem ser simplificadas com a remoção da variável $A$!
  }
%- - - - - - - - - - - - - - - - - - - - - - - - - - - - - - - - 
  \item  \textcolor{red}{Gramática $G^2_{\myling}$ não ambígua que gera as cadeias da linguagem $\mathcal{L}_{\myling}$:\\ $G_{\myling}=(V,\Sigma,P,S)=(\{S\},\{(,)\},P,S)$, com
  $$
   P =
   \left\{\begin{array}{l}
    S \to (S)S \mid \varepsilon
   \end{array}\right\}.
  $$
  }
\end{itemize}
%}
%
%
\noindent\textbf{Obs.:} Esta observação e os comandos de cor (``\verb|\textcolor{}{}|'') podem ser removidos da versão a ser submetida na plataforma Turing.
\end{document}
